\documentclass{article}
\usepackage{hyperref}
\usepackage{xcolor}

\title{2-D video game documentation:\\Name to be defined}
\date{\today}
\author{cal-rmedina}

\begin{document}
  \maketitle

\textbf{Runs on:} Windows, Linux and Mac.
\textbf{Website:} \href{https://godotengine.org/}{Godot}.
\textbf{YouTube Chanel:} \href{https://www.youtube.com/@GodotEngineOfficial/featured}{YouTube chanel}.


The following chapters desctibe the structure of the video game; character devlopment, playability, style and animation.

\section{Motivation}
\section{Color palette}
\section{Shapes and borders of the animation}
\section{Sound atmosphere}
\section{Physics}

This chapter includes the physical concepts used inside the video game, brief explanation and equations used.

\subsection{Nuclear decay}

\subsection{Ballistic movement of the oponents}

For the moving enemies, a random ballistic motion has been used, it is defined as follows:

\begin{equation}
\bar{r}_i(t+\delta t) = \bar{r}_i(t) + \bar{v}\delta t\times n_{rnd}
\end{equation}

where $\bar{r}_i(t)$ is the position of enemy $i$ at time $t$, $\bar{v}$ is the
constant velocity of the particles and $n_{rdn}$ is an uniform random number
between [0,1]. {$n_{rdn}$} defines random path of the enemy. If it is not
included, the movement becomes ballistic.

\subsection{Boundaries}

The battle field is defined as a rectangle, once the enemy interacts with the boundaries of the maps, a bouncing rule is defined, in our case \texttt{bouncing back rule}.








\end{document}
