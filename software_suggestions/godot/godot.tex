\documentclass{article}
\usepackage{hyperref}
\usepackage{xcolor}

\title{Godot Engine}
\date{\today}
\author{cal-rmedina}

\begin{document}
  \maketitle

\textbf{Runs on:} Windows, Linux and Mac.
\textbf{Website:} \href{https://godotengine.org/}{Godot}.
\textbf{YouTube Chanel:} \href{https://www.youtube.com/@GodotEngineOfficial/featured}{YouTube chanel}.

After quickly exploring Godot Engine, I've decided that Godot has the features
needed to continue developing the game.

Among its advantages it includes:

\begin{enumerate}
  \item It has its own graphical environment plus the possibility to add animations
from Blender.
  \item It could be easily ported to multiple OS (Windows, Linux, Mac, Android,
Web, etc.). 
  \item It allows using different programming languages, moreover, it has its
own programming language \href{https://gdscript.com}{GDScript}.
  \item It has enough documentation, webpages and community to develop owr ideas.
\end{enumerate}

\textcolor{red}{TODO list:}

\begin{itemize}
  \item Make the \href{https://gdquest.github.io/learn-gdscript}{GDScript
tutorial}, it is in English and Spanish.
  \item Quick review of the \href{https://docs.godotengine.org/en/stable}{documentation}.
  \item Download/Install Godot Engine on your OS.
\end{itemize}

\end{document}
