\documentclass{article}
\usepackage{hyperref}

\title{Ideas for the plot of the RPG}
\date{\today}
\author{cal-rmedina}

\begin{document}
  \maketitle

All the RPGs need the basic features like modifying the main character, like
selecting name, gender, etc.

\section{Alice in Quantumland: An allegory of quantum physics.}

Develop the map completely based on a Quantum world; the puzzles to solve could
be quantum phenomena (entanglement, superposition, etc.), it could be fully
based on [\cite{gilmore_2010}], taking the charachters from the book, we could
also take as characters of the map the developers of Quantum Mechanics (Plank,
Heisenberg, Schr{\"o}dinger, etc) [\cite{volpi_2016},\cite{cline_1995}],
there's is also a TV-Serie call \href{https://en.wikipedia.org/wiki/Devs}{Devs}
which plot takes topics of the Quantum computing revolution.

\subsection{Details}

\begin{itemize}
  \item Check the basic phenomena to be included in the story to guaranty its
correct implementation.
  \item Use the quantum mechanics paradoxes as part of the maps.
  \item Quantum Computing, Nuclear Physics, High Energy physics could be used
in subsequnent levels.
\end{itemize}

\section{Do robots dream of electric sheep?}

The idea for this topic is to develop a game where the main character asks
question across the map to find clues and determine who the android(s) is/are,
among the members of the community; design several clues and puzzles to be
solved, the main idea is to apply the
\href{https://en.wikipedia.org/wiki/Turing_test}{Turing test}, concepts like
consciousness, AI can be devolped. The main ideas could be taken from the
TV-Serie call \href{https://en.wikipedia.org/wiki/Mr._Robot}{Mr. Robot} or
[\cite{penrose_2016} \cite{dick_1975}]

\subsection{Details}

\begin{itemize}
  \item Generate a number of puzzles to identify who the android is.
  \item Randomize the map to put the characters in different location each time.
\end{itemize}

\bibliography{bibtex} 
\bibliographystyle{ieeetr}


\end{document}
